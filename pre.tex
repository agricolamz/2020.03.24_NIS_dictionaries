\documentclass[xcolor=table]{beamer}
\usetheme{default}
\definecolor{darkscarlet}{rgb}{0.34, 0.01, 0.1}
\usecolortheme{default}
\usecolortheme[named=darkscarlet]{structure}
\setbeamercolor{title}{bg=white, fg=darkscarlet}
\definecolor{cerise}{rgb}{0.87, 0.19, 0.39}
\hypersetup{colorlinks=TRUE,linkcolor=cerise,urlcolor=cerise, citecolor=cerise}

% nummering

\setbeamertemplate{caption}[numbered]

% taal

% tekens
\usepackage{fontspec}
\usefonttheme{serif}
\setmainfont[BoldFont=brillb.ttf, ItalicFont=brilli.ttf, BoldItalicFont=brillbi.ttf]{brill.ttf}

% tekst doorstrepen, kennelijk heb je daar een apart pakket voor nodig

\usepackage[normalem]{ulem}

% plaatjes

\usepackage{graphicx}
\usepackage{subfig}

% tabellen

\usepackage{multirow}

% voorbeelden

\usepackage{philex}
\usepackage{forest}

% glossen

\usepackage{leipzig}
\newleipzig{hab}{hab}{habitual}
\newleipzig{aor}{aor}{aorist}
\makeglossaries

% bibliography

\usepackage[backend=biber,
        bibstyle=biblatex-sp-unified,
        citestyle=sp-authoryear-comp,
        doi=false,
        maxcitenames=3,
        maxbibnames=99]{biblatex}
\addbibresource{bibliography.bib}


% custom footline

\setbeamerfont{footline}{size=\fontsize{20}{20}\selectfont}

\newcommand{\Ffootline}{\footnotesize
\insertsection
\hfill
\href{https://github.com/agricolamz/2020.03.24_NIS_dictionaries}{github/agricolamz/2020.03.24\_NIS\_dictionaries}
\hfill
\insertframenumber/\inserttotalframenumber} 

% custom footline deel 2

\setbeamertemplate{footline}{%
\usebeamerfont{structure}
\begin{beamercolorbox}[wd=\paperwidth,ht=2.25ex,dp=1ex]{title in head/foot}%
\Tiny\hspace*{4mm} \Ffootline \hspace{4mm}
\end{beamercolorbox}}

% navigatiesymbolen uitzetten

\beamertemplatenavigationsymbolsempty
 
 
% titelpagina 

\title{Phonological and morphological variation in Botlikh: \\ Comparing two dictionaries}
\author{George Moroz, Chiara Naccarato, Samira Verhees \\
Linguistic Convergence Laboratory, NRU HSE}
\date{24.03.2020}

% links

\usepackage{hyperref}

%\newcommand\pro{\item[$+$]}
%\newcommand\con{\item[$-$]}

% ž š šč č `