\input{pre} 

\begin{document}

\begin{frame}
\titlepage
\end{frame}

\section{Introduction}
\begin{frame}{Botlikh}
    \begin{itemize}
        \item Botlikh < Andic < Avar-Andic-Tsezic < East Caucasian
        \item Spoken by \textasciitilde{}5,000-8,000 speakers
        \item Three villages in the Botlikh district of the Republic of Daghestan: Botlikh, Miarso, and Ashino
        \item Unwritten and mostly spoken at home; the Cyrillic script of Avar functions as an ad hoc writing system on social media
        \item Evaluated as ``threatened'' by Ethnologue \citep{simonsfenning2018}, but many children still speak the language and attitudes are positive
        \item Heavy influence from Avar and Russian 
    \end{itemize}
\end{frame}

\begin{frame}{Botlikh}
\begin{figure}[h]
\centering
\fbox{\includegraphics[scale=0.4]{images/globalmap.png}}
\caption{Botlikh on the map}
\end{figure}
\end{frame}

\begin{frame}{Botlikh literature}
\begin{itemize}
    \item One full reference grammar in Georgian \citep{gudava1962}
    \item Several short sketches mostly based on information contained in the grammar by Togo E. Gudava: \citep{gudava1967, azaev2000, saidova2001, magomedbekova2001, xalidova2017, alekseevverhees}
    \item Several works on the lexicon and word formation \citep{azaev1975, sulejmanova2013, alekseev2016}
    \item In general poorly described compared to other Andic languages like Godoberi or Bagvalal, BUT two Botlikh-Russian dictionaries are available to date \citep{saidovaabusov2012, alekseev2019}
\end{itemize}
\end{frame}

\begin{frame}{Two dictionaries}
\begin{figure}[h]
\begin{subfigure}
\includegraphics[height=6cm]{images/abusov2012.jpg} 
\label{abusov2012}
\end{subfigure}
\begin{subfigure}
\includegraphics[height=6cm]{images/alekseev2019.jpg}
\label{alekseev2019}
\end{subfigure}
\label{two_dicts}
\caption{Two Botlikh-Russian dictionaries}
\end{figure}
\end{frame}

\begin{frame}{Two dictionaries}
\begin{itemize}
    \item \citep{saidovaabusov2012} compiled in the 2000s by a native speaker of Botlikh (Magomed G. Abusov) and an experienced linguist (Patimat A. Saidova)
    \item \citep{alekseev2019} compiled in the 1960s/1970s by a native speaker of Botlikh and philologist (Xalil G. Azaev), later (in the 2000s) systematized by an experienced linguist (Mixail E. Alekseev), and published posthumously after the editing by Timur A. Maisak
\end{itemize}
\end{frame}

\begin{frame}{Two dictionaries}
\begin{itemize}
    \item Comparable both quantitatively and qualitatively
    \item \textasciitilde{}8,000 headwords for \citep{saidovaabusov2012} vs. \textasciitilde{}9,000 words and expressions for \citep{alekseev2019}
    \item Although the data in \citep{alekseev2019} were collected several decades earlier, Magomed G. Abusov also consulted elderly speakers with the aim of collecting archaic vocabulary 
    \item \citep{saidovaabusov2012} also contains some notes on Miarso; reference to Miarso variants is not explicit in \citep{alekseev2019}, but it seems that such variants are occasionally reported in this dictionary too
    \item No metadata on the speakers consulted
    \item At first glance, the two resources seemed to display variation
\end{itemize}
\end{frame}

\begin{frame}{Our research}
\begin{itemize}
    \item Comparison of the two resources
    \item A unique opportunity to conduct a quantitative investigation of an understudied language
    \item Provide numerical approximations for the impressionistic observations available in the existing literature
    \item Analysis of both phonological and morphological features 
    \item Detect patterns of systematic variation within these two areas
\end{itemize}
\end{frame}

\begin{frame}{Outline}
\begin{itemize}
    \item Data 
    \begin{itemize}
        \item merging
        \item extracting grammatical information
        \item pairing and annotation
    \end{itemize}
    \item Analysis
    \begin{itemize}
        \item phonology (George Moroz)
        \item nominal morphology (Chiara Naccarato)
        \item verbal morphology (Samira Verhees)
    \end{itemize}
    \item Results and discussion
    \item Methodological remarks 
\end{itemize}
\end{frame}

\section{Data}
\begin{frame}{Merging}
From two .doc files to one .xls through a painstaking process of unification (George Moroz)
\begin{figure}[h]
\centering
\fbox{\includegraphics[scale=0.17]{images/dicts.jpg}}
\caption{Merging the dictionaries}
\end{figure}
\end{frame}

\begin{frame}{Extracting grammatical information}
\begin{itemize}
    \item Total number of lexemes extracted: 8,464 from \citep{saidovaabusov2012} and 6,821 from \citep{alekseev2019}
    \item Nouns: 2,871 from \citep{saidovaabusov2012} and 3,097 from \citep{alekseev2019}
    \begin{itemize}
        \item grammatical information: genitive and plural
    \end{itemize}
    \item Verbs: 1,504 from \citep{saidovaabusov2012} and 1,640 from \citep{alekseev2019}
    \begin{itemize}
        \item grammatical information: habitual and aorist
    \end{itemize}
\end{itemize}
\end{frame}

\begin{frame}{Extracting grammatical information}
\begin{figure}[h]
\centering
\fbox{\includegraphics[scale=0.4]{images/table.jpg}}
\caption{The database}
\end{figure}
\end{frame}

\begin{frame}{Pairing}
\begin{itemize}
    \item We manually checked for lexemes represented in both dictionaries to carry out phonological and morphological analysis 
    % Here we could show Garik's Eulero Venn diagram, but we should check the numbers because I have the impression that we made quite a mess in our database  
\end{itemize}
\end{frame}

\begin{frame}{Annotation}
\begin{itemize}
    \item Manual correction of automatically extracted information about grammatical features
    \item Addition of further annotation (for features that appeared to be potentially relevant for our research):
    \begin{itemize}
        \item masdars
        \item borrowings
    \end{itemize}
\end{itemize}
    
\end{frame}

\section{Phonology}
\begin{frame}{Phonology}
    % Garik's research
\end{frame}

\section{Nominal morphology}
\begin{frame}{Nominal morphology}
Two topics investigated:
\begin{itemize}
    \item Formation of the plural
    \begin{itemize}
        \item to check the productivity of different suffixes
    \end{itemize}
    \item Formation of the genitive
    \begin{itemize}
        \item to study alternations in the formation of oblique stems
    \end{itemize}
\end{itemize}
Comparison of the two resources to look for possible variation in such areas of nominal morphology \\ (based on 1,072 pairs retrieved during the first annotation round)
\end{frame}

\begin{frame}{Plural formation in Botlikh}
\begin{itemize}
    \item A suffix is attached to the absolutive stem: \\ \textit{na} `thing' < \textit{na-\textbf{baɬi}} `things'
    \item With stems ending in a consonant, the vowel \textit{-a-} is often inserted before the suffix: \\ \textit{majmalak}  `monkey' < \textit{majmalak-\textbf{a}-\textbf{baɬi}} `monkeys'
    \item With stems ending in a vowel, alternation can occur: \\ \textit{ruš\textbf{a}}  `tree' < \textit{ruš\textbf{i}-\textbf{baɬi}} `trees', \\ \textit{sal\textbf{u}}  `tooth' < \textit{sal\textbf{a}-\textbf{baɬi}} `teeth', \\ \textit{buraɬ\textbf{i}}  `pitcher' < \textit{buraɬ\textbf{a}-\textbf{baɬi}} `pitchers'
\end{itemize}
\end{frame}

\begin{frame}{Plural formation in Botlikh}
Among the most common suffixes are:
\begin{itemize}
    \item \textit{-baɬi} and allomorphs (\textit{-maɬi} for stems ending in a nasal, \textit{-wabaɬi} for stems ending in -\textit{u}, etc.), the variant \textit{-zabaɬi} (mostly with borrowings) \\ \textit{apicer} `officer' < \textit{apicer-\textbf{zabaɬi}} `officers'  
    \item \textit{-de} (mostly for stems ending in a sonorant) \\ \textit{ambur} `roof' < \textit{ambur-\textbf{de}} `roofs'
    \item \textit{-e} and its variant \textit{-we} (for stems ending in -\textit{u}) \\ \textit{čan} `deer' < \textit{čan-\textbf{e}} `deers'
\end{itemize}
Other, less common, suffixes are: -\textit{(b)daɬi}, -\textit{(b)diɬi}, -\textit{(a)l}, -\textit{rdi}, -\textit{bala(l)}
\end{frame}

\begin{frame}{Plural formation in Botlikh}
\centering
Can our dictionary data help us be more precise about the distribution/frequency/productivity of plural suffixes in Botlikh? \\ ... \\ Do the two dictionaries show any variation in these respects?
\end{frame}

\begin{frame}{Plural suffixes in the dictionaries}
\begin{itemize}
    \item Plural suffixes are not reported for all nouns, cf. \textit{singularia tantum} and plural entries (nationalities, \textit{pluralia tantum})
    \item Quite often more than one variant is reported
\end{itemize}
\begin{table}[]
\caption{Plural suffixes in the dictionaries}
\centering
\begin{tabular}{lcc}
          & \multicolumn{1}{l}{\textbf{Saidova \& Abusov (2012)}} & \multicolumn{1}{l}{\textbf{Alekseev \& Azaev (2019)}} \\
\textbf{-\textit{(x)baɬi}}  & 292                                          & 298                                          \\
\textbf{-\textit{de}}       & 128                                          & 354                                          \\
\textbf{-\textit{(w)e}}     & 141                                          & 239                                          \\
\textbf{other}     & 24                                           & 21                                           \\
\textbf{no plural} & 499                                          & 193                                         
\end{tabular}
\end{table}
\end{frame}

\begin{frame}{Plural suffixes in the dictionaries}
\begin{figure}[h]
\caption{Plural suffixes in the dictionaries}
\centering
\includegraphics[height=5.5cm]{images/plural.png}
\end{figure}
\centering
\small (X-squared = 47.118, df = 2, p-value = 5.869e-11)
\end{frame}

\begin{frame}{Plural suffixes in the dictionaries}
\begin{itemize}
    \item Preference for -\textit{(x)baɬi} over -\textit{de} in \citep{saidovaabusov2012} vs. the opposite trend in \citep{alekseev2019}
    \item The higher frequency of -\textit{de} in \citep{alekseev2019} is partly due to masdars
    \begin{itemize}
        \item \citep{saidovaabusov2012} almost never report the plural form for such nouns, whereas \citep{alekseev2019} consistently report -\textit{de}
    \end{itemize}
    \item Variation often involves (but is not restricted to) borrowings
    \begin{itemize}
        \item \textit{birgadir} `foreman' < \textit{birgadir-\textbf{zabaɬi}} vs. \textit{birgadir-\textbf{de}}
        \item \textit{kassir} `cashier' < \textit{kassir-\textbf{zabaɬi}} vs. \textit{kassir-\textbf{de}}
    \end{itemize}
    \item The frequent mentioning of more than one variant might suggest idiosyncratic variation
\end{itemize}
\end{frame}

\begin{frame}{Case declension in Botlikh}
Two declension types
\begin{itemize}
    \item I type --- the stem does not change when a suffix is attached (mostly stems ending in a vowel and masdars) \\ \textit{bab\textbf{u}} `mom' < \textit{bab\textbf{u}-\textbf{ɬi}} (genitive) \\ \textit{masir} `measurement' < \textit{masir-\textbf{ɬi}} (genitive)  
    \item II type --- case suffixes are attached to the oblique stem of the noun (mostly stems ending in a consonant, sometimes stems ending in a vowel) \\ \textit{askar} `army' < \textit{askar-\textbf{a}-\textbf{ɬi}} (genitive) \\ \textit{din} `religion' < \textit{din-\textbf{i}-\textbf{ɬi}} (genitive) \\ \textit{ima} `father' < \textit{im\textbf{u}-\textbf{ɬi}} (genitive) 
\end{itemize}
\end{frame}

\begin{frame}{Case declension in Botlikh}
\centering
Can our dictionary data help us be more precise about the patterns of case declension (oblique stem formation) and their frequencies? \\ ... \\ Do the two dictionaries show any variation in these respects?    
\end{frame}

\begin{frame}{Oblique stem formation in the dictionaries}
We used the grammatical information included in the dictionaries (genitive suffix) to investigate oblique stem formation in Botlikh
\begin{table}[]
\caption{Oblique stem formation in the dictionaries}
\centering
\begin{tabular}{|c|c|c|c|c|c|c|c|c|}
\hline
            & \multicolumn{2}{c|}{\textit{-ɬi}}                       & \multicolumn{2}{c|}{\textit{-\textbf{a}-ɬi}}                                       & \multicolumn{2}{c|}{\textit{-\textbf{i}-ɬi}}                                       & \multicolumn{2}{c|}{\textit{-\textbf{u}-ɬi}}                   \\ \hline
consonant   & {\color[HTML]{1B9045} 232} & {\color[HTML]{CE6301} 266} & {\color[HTML]{1B9045} \textbf{228}} & {\color[HTML]{CE6301} \textbf{167}} & {\color[HTML]{1B9045} \textbf{104}} & {\color[HTML]{CE6301} \textbf{141}} & {\color[HTML]{1B9045} -}  & {\color[HTML]{CE6301} 1}  \\ \hline
\textit{-a} & {\color[HTML]{1B9045} 182} & {\color[HTML]{CE6301} 167} & {\color[HTML]{1B9045} -}            & {\color[HTML]{CE6301} -}            & {\color[HTML]{1B9045} \textbf{3}}   & {\color[HTML]{CE6301} \textbf{13}}  & {\color[HTML]{1B9045} 15} & {\color[HTML]{CE6301} 14} \\ \hline
\textit{-i} & {\color[HTML]{1B9045} 143} & {\color[HTML]{CE6301} 151} & {\color[HTML]{1B9045} \textbf{10}}  & {\color[HTML]{CE6301} \textbf{4}}   & {\color[HTML]{1B9045} -}            & {\color[HTML]{CE6301} -}            & {\color[HTML]{1B9045} 3}  & {\color[HTML]{CE6301} 2}  \\ \hline
\textit{-u} & {\color[HTML]{1B9045} 81}  & {\color[HTML]{CE6301} 78}  & {\color[HTML]{1B9045} 1}            & {\color[HTML]{CE6301} -}            & {\color[HTML]{1B9045} -}            & {\color[HTML]{CE6301} 3}            & {\color[HTML]{1B9045} -}  & {\color[HTML]{CE6301} -}  \\ \hline
\textit{-e} & {\color[HTML]{1B9045} 6}   & {\color[HTML]{CE6301} 6}   & {\color[HTML]{1B9045} -}            & {\color[HTML]{CE6301} -}            & {\color[HTML]{1B9045} -}            & {\color[HTML]{CE6301} -}            & {\color[HTML]{1B9045} -}  & {\color[HTML]{CE6301} -}  \\ \hline
\textit{-o} & {\color[HTML]{1B9045} 7}   & {\color[HTML]{CE6301} 7}   & {\color[HTML]{1B9045} -}            & {\color[HTML]{CE6301} -}            & {\color[HTML]{1B9045} -}            & {\color[HTML]{CE6301} -}            & {\color[HTML]{1B9045} -}  & {\color[HTML]{CE6301} -}  \\ \hline
\end{tabular}
\end{table}
\centering
\small {\color[HTML]{1B9045} Saidova \& Abusov (2012)} vs. {\color[HTML]{CE6301} Alekseev \& Azaev (2019)}
\end{frame}

\begin{frame}{Oblique stem formation in the dictionaries}
\begin{figure}[h]
\centering
\caption{Oblique stem formation for stems ending in a consonant}
\includegraphics[height=5.5cm]{images/genitive.png}
\end{figure}
\centering
\small (X-squared = 13.523, df = 1, p-value = 0.0002357)
\end{frame}

\begin{frame}{Oblique stem formation in the dictionaries}
\begin{itemize}
    \item Significant variation between the two dictionaries in the formation of oblique stems for nouns ending in a consonant
    \item This again involves (but is not restricted to) borrowings (a general preference for -\textit{a}- over -\textit{i}- in \citep{saidovaabusov2012})
    \begin{itemize}
        \item \textit{dakument} `document' < \textit{dakument-\textbf{a}-ɬi} vs. \textit{dakument-\textbf{i}-ɬi}
        \item \textit{kassir} `cashier' < \textit{kassir-\textbf{a}-ɬi} vs. \textit{kassir-\textbf{i}-ɬi} 
        \item \textit{adijal} `blanket' < \textit{adijal-\textbf{a}-ɬi} vs. \textit{adijal-\textbf{i}-ɬi} 
    \end{itemize}
    \item Different variants for one and the same noun are reported far less frequently as compared to plural suffixes
\end{itemize}
\end{frame}

\section{Verbal morphology}
\begin{frame}{Verbal morphology}
Formation of present (habitualis) and past (aorist) forms of:
\begin{itemize}
    \item Basic verbs (infinitive in -\textit{i})
    \item Derived verbs (infinitive in -\textit{ɬi})
    \item Causative verbs (infinitive in -\textit{a-j})
\end{itemize}
Comparison of the two resources to look for possible variation in such areas of verbal morphology \\ (based on 554 pairs retrieved during the first annotation round)    
\end{frame}

\begin{frame}{Basic verbs}
\begin{itemize}
    \item Habitualis: -\textit{e}
    \item Aorist: -\textit{a} / -\textit{u} / -\textit{iw}
\end{itemize}
\begin{table}[]
\caption{Basic verbs: inflection}
\centering
\begin{tabular}{l|l|l|l}
        & Infinitive       & Habitualis       & Aorist            \\ \hline
see     & \textit{haʁ-i}   & \textit{haʁ-e}   & \textit{haʁ-\textbf{a}}    \\
do      & \textit{ih-i}    & \textit{ih-e}    & \textit{ih-\textbf{u}}     \\
be able & \textit{bažar-i} & \textit{bažar-e} & \textit{bažar-\textbf{iw}}
\end{tabular}
\end{table}
\end{frame}

\begin{frame}{Basic verbs}
\begin{figure}[h]
\centering
\caption{Aorist suffixes in the dictionaries}
\includegraphics[scale=0.5]{images/pst.png}
\end{figure}
\end{frame}

\begin{frame}{Derived verbs}
Analytic formation of both the habitualis and the aorist with auxiliaries
\begin{itemize}
    \item be: \textit{b-uk'-e}, \textit{b-uk'-a}
    \item become: \textit{b-ah-e}, \textit{b-ah-u}
\end{itemize}
\begin{table}[]
\caption{Derived verbs: inflection}
\centering
\begin{tabular}{l|l|l|l}
      & Infinitive         & Habitualis             & Aorist                 \\ \hline
roar  & \textit{buda-ɬi}   & \textit{buda \textbf{b-uk'-e}}  & \textit{buda \textbf{b-uk'-a}}  \\
bleat & \textit{baʕada-ɬi} & \textit{baʕada \textbf{b-ah-e}} & \textit{baʕada \textbf{b-ah-u}}
\end{tabular}
\end{table}
\end{frame}

\begin{frame}{Causative verbs}
\begin{itemize}
    \item -\textit{o} < \textit{*-a-u} [\textsc{-caus-aor}]
    \item -\textit{mal-e} a reduced form of -\textit{malih-e}?
    \item -\textit{mal-o} rarely found in the data
\end{itemize}
\begin{table}[]
\caption{Causative verbs: inflection}
\centering
\begin{tabular}{l|l|l|l}
         & Infinitive        & Habitualis                & Aorist                    \\ \hline
resettle & \textit{guč-a-j}  & \textit{guč-\textbf{e}}            & \textit{guč-\textbf{o}}            \\
roast    & \textit{žad-a-j}  & \textit{žad-a-j-\textbf{mal-e}}    & \textit{žad-a-j-\textbf{mal-o}}    \\
sew up   & \textit{mik'-a-j} & \textit{mik'-a-j-\textbf{mal-e}}   & \textit{mik'-\textbf{o}}           \\
sew up   & \textit{mik'-a-j} & \textit{mik'-a-j-\textbf{malih-e}} & \textit{mik'-a-j-\textbf{malih-u}}
\end{tabular}
\end{table}
\end{frame}

\begin{frame}{Habitualis in the dictionaries}
\begin{figure}[h]
\centering
\caption{Habitualis}
\includegraphics[scale=0.45]{images/infxprs.png}
\end{figure}
\end{frame}

\begin{frame}{Aorist in the dictionaries}
\begin{figure}[h]
\centering
\caption{Aorist}
\includegraphics[scale=0.45]{images/infxpst.png}
\end{figure}
\end{frame}

\begin{frame}{Variation}
\begin{itemize}
    \item Basic verbs display little variation in the aorist (and no variation at all in the habitualis)
    \item Most variation is observed for derived and causative verbs:
    \begin{itemize}
        \item Derived verbs: preference for auxiliary `be' in \citep{saidovaabusov2012} vs. `become' in \citep{alekseev2019} \\ BUT it seems that this is just a matter of personal taste for citation forms: examples in the dictionary entries show that both variants are possible
        \item Causative verbs: full (older?) forms -\textit{malih-e} and -\textit{malih-u} in \citep{alekseev2019} vs. reduced form -\textit{mal-e} and synthetic -\textit{o} in \citep{saidovaabusov2012} \\ This might be interpreted as diachronic variation, since the data in \citep{alekseev2019} are older
    \end{itemize}
\end{itemize}
\end{frame}

\section{Discussion}
\begin{frame}{Discussion}
    
\end{frame}

\section{Methodological remarks}
\begin{frame}{Methodological remarks}
    
\end{frame}

\section{The end}
\begin{frame}{The end}
\begin{figure}[h]
\centering
\fbox{\includegraphics[height=6cm]{images/arqule.jpg}}
\end{figure}
\end{frame}

\section{Abbreviations}
\begin{frame}{Abbreviations}
\tiny{\printglossary}
\end{frame}

\section{References}
\begin{frame}[allowframebreaks]{References}
\printbibliography
\end{frame}

\end{document}